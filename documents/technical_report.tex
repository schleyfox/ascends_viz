\documentclass[12pt,letterpaper]{report}
\usepackage[pdftex]{graphicx}
\usepackage[top=1in,bottom=1in,left=1in,right=1in]{geometry}
%\usepackage{fancyhdr}

%\setlength{\headheight}{15pt}
%\pagestyle{fancy}

\newcommand{\HRule}{\rule{\linewidth}{0.5mm}}

\begin{document}
 \begin{titlepage}

\begin{center}

\includegraphics[width=0.15\textwidth]{./logo}\\[1cm]

\textsc{\LARGE DEVELOP}\\[1.5cm]

\textsc{\Large NASA Langley Research Center}\\[0.5cm]

\HRule \\[0.4cm]
{ \huge \bfseries ASCENDS Data Visualization}\\[0.4cm]

\HRule \\[1.5cm]

\begin{minipage}{0.4\textwidth}
\begin{flushleft} \large
\emph{Principle Investigators:}\\
T. Nelson \textsc{Hillyer}\\
Jacob \textsc{Atkins}\\
Benjamin S. \textsc{Hughes}\\
Andrew \textsc{Kramolisch}\\
Mark \textsc{Lotts}
\end{flushleft}
\end{minipage}{0.4\textwidth}
\begin{flushright} \large
\emph{Science Advisors:}\\
Dr. F. Wallace \textsc{Harrison}\\
Dr. Edward \textsc{Browell}
\end{flushright}
\end{minipage}

\vfill

{\large \today}

\end{center}

\end{titlepage}



 \begin{abstract}
 The Active Sensing of $CO_2$ Emissions Over Nights, Days, and Seasons (ASCENDS) mission aims to enhance understanding of $CO_2$ in the global carbon cycle with three scientific objectives: quantifying global spatial distributions of atmospheric $CO_2$ on scales of weather models in the 2010--2020 era, quantifying current distributions of terrestrail and oceanic sources and sinks of $CO_2$ on $1^\circ\times1^\circ$ grids at weekly resolution, and providing a scientific basis for future $CO_2$ projections of sources and sinks through data-driven enhancements of Earth system process modeling.  The use of ruby and Google Earth can further prove the utility of ASCENDS in $CO_2$ pollution sources and sinks in the United States.
 \end{abstract}

 \section*{Introduction}
  \subsection*{ASCENDS Mission}
   \paragraph{}
    The Active Sensing of Carbon Dioxide Emissions over Nights, Days, and Seasons (ASCENDS) mission is currently an exploratory mission using simultaneous laser remote sensing of $CO_2$ and $O_2$. The technology employed was based on a recommendation from the National Research Council of the United States National Academies. Active sensing allows the ASCENDS team to enhance understanding of the role of $CO_2$ in the global carbon cycle, due to its lack of seasonal, latitudinal, or diurnal bias.
  \subsection*{Visualization}
   \paragraph{}
    The ASCENDS Team project involved visualizing the data gathered with active sensing technology. Whereas a barrage of numbers is overwhelming, visualizations allow for clarity with a basic understanding of the data. Additionally, visualizations can display an enormous amount of data in an easily accessible format. ASCENDS technology is extremely accurate, allowing for visuals to compare and contrast with other $CO_2$ models.
   \paragraph{}
    Generation of the visualizations is accomplished by first piecing the data together in a relational database using the Gigantron framework. Gigantron is an in-house framework designed to ease the process of importing data, linking it together, and manipulating it. Gigantron runs on top of JRuby, a Ruby implementation on the Java Virtual Machine, for speed and rapid development. Rake tasks inside the generated Gigantron project allow for the dynamic generation of Keyhole Markup Language (KML) files, by utilizing ActiveRecord and ruby-kml.
  \subsection*{Importance}
   \paragraph{}
    The ability of visualization to be utilized and promote the better understanding of $CO_2$ movement patterns explicates the necessity for an ASCENDS project satellite. More data recorded and imported into the visualization will eventually allow for a global analysis of atmospheric $CO_2$ levels. The data from the HYSPLIT model allowed for the visualization of carbon dioxide plumes' back trajectories, thus allowing for the study of the movement and sources of carbon dioxide. In conjunction with the in-situ data, the ITT data showed the carbon dioxide mixing ratio with the wind. The differences between the observed data and the predicted data permitted an exceptionally high-resolution analysis of the carbon dioxide levels. The predictive models closely resembled the observed data, implying that the US will be a net source of $CO_2$.
 \section*{Methodology}
  \subsection*{Data Obtainment}
   \paragraph{}
    The first step in the project was to obtain raw data from the ASCENDS team's various flights in a plane outfitted with $CO_2$ sensing equipment over eastern Virginia and North Carolina.  The data was provided in multiple spreadsheets and raw text files, and included information about readings such as wind speed, direction, latitude, longitude, CO2, and date. Time stamps connected the data across multiple files. One set of data contained the ITT information, the average $CO_2$ level over the column of air below the aircraft. The second set of data is the in-situ data, the mixing ratio of $CO_2$ as compared to the rest of the air.
  \subsection*{Data Visualization}
   \paragraph{}
    Once the data files were obtained, scripts written in the JRuby to analyze and parse essential information. The resulting data was then written into a KML file using the ruby-kml library. The algorithms in the program were used to display the ITT data in column format, with the in-situ data on top of the ITT columns. A separate algorithm was used to color-code the resulting polygonal columns according to the level of $CO_2$ in the air, using the established ASCENDS color bar. The resulting KML file was imported into Google Earth, where the latitude, longitude, and altitude data was used to map out the columns on the exact place where the data points were gathered. This allows the data to be analyzed in both a geographical and visual sense.
  \subsection*{Model Visualization and Comparison}
   \paragraph{}
    Data from various predictive models was written into KML files and imported into Google Earth to be compared with the observed data. A cross-sectional analysis of the observed data versus the predictive data was visualized based on both the time and location of the data point.
 \section*{Results}
  \subsection*{Visualization}
  \subsection*{Carbon Levels}

 \section*{Discussion}
  \paragraph{}
   Carbon dioxide in the atmosphere can be visualized according to the exact geographic location. This allows both scientists and politicians to better interpret the visualized data. The display of in-situ mixing ratio data alongside the ITT data allows for a more in-depth analysis of how carbon dioxide levels affect the composition of the air in specific atmospheric locations. This ASCENDS visualization will be available to use to calibrate future visualizations and serve as a benchmark for $CO_2$ analysis in further projects.
 \section*{Conclusion}
  \paragraph{}
   As global warming and carbon dioxide management directly affect the future of the Earth, air pollution is becoming an increasingly relevant topic.  This project is important because it allows for the advancement of environmental research as well as hopefully leading to scientific breakthroughs that will help the environment's long-term sustainability.
  \paragraph{}
   The visualization will ultimately help the ASCENDS mission in obtaining funding from the government.  An ASCENDS satellite will allow for global carbon dioxide atmospheric studies.  High resolution carbon dioxide research will now be possible, thus helping in the fight against pollution and global warming.  The visualization could be improved by adding more carbon dioxide predictive models, so that the observed data can becompared to multiple expected values.  A more friendly user interface would also make the visualization more accessible to people who have minimal experience with Google Earth and KML files.  Optimizing the visualization algorithms will also improve the product and it will then be less computationally demanding.  This visualization must be used by the scientific community to aid research and eventually discover solutions to the world's environmental problems.

 \section*{Acknoledgements}

\end{document}


