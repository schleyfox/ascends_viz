\documentclass[12pt,letterpaper]{report}
\usepackage[pdftex]{graphicx}
\usepackage[top=1in,bottom=1in,left=1in,right=1in]{geometry}
%\usepackage{fancyhdr}

%\setlength{\headheight}{15pt}
%\pagestyle{fancy}

\newcommand{\HRule}{\rule{\linewidth}{0.5mm}}

\begin{document}
 \begin{titlepage}

\begin{center}

\includegraphics[width=0.15\textwidth]{./logo}\\[1cm]

\textsc{\LARGE DEVELOP}\\[1.5cm]

\textsc{\Large NASA Langley Research Center}\\[0.5cm]

\HRule \\[0.4cm]
{ \huge \bfseries ASCENDS Data Visualization}\\[0.4cm]

\HRule \\[1.5cm]

\begin{minipage}{0.4\textwidth}
\begin{flushleft} \large
\emph{Principle Investigators:}\\
T. Nelson \textsc{Hillyer}\\
Jacob \textsc{Atkins}\\
Benjamin S. \textsc{Hughes}\\
Andrew \textsc{Kramolisch}\\
Mark \textsc{Lotts}
\end{flushleft}
\end{minipage}{0.4\textwidth}
\begin{flushright} \large
\emph{Science Advisors:}\\
Dr. F. Wallace \textsc{Harrison}\\
Dr. Edward \textsc{Browell}
\end{flushright}
\end{minipage}

\vfill

{\large \today}

\end{center}

\end{titlepage}



 \begin{abstract}
The Active Sensing of $CO_2$ Emissions Over Nights, Days, and Seasons (ASCENDS) misson is to enhance understanding of $CO_2$ in the global carbon cycle with three scientific objectives: quantifying global spatial distributions of atmospheric $CO_2$ on scales of weather models in the 2010--2020 era, quantifying current distributions of terrestrial and oceanic sources and sinks of $CO_2$ on $1^\circ \times 1^\circ$ grids at weekly resolution, and provide a scientific basis for future $CO_2$ projections of sources and sinks through data-driven enhancements of Earth system process modeling.  The use of Ruby and Google Earth can further prove the utility of ASCENDS in locating $CO_2$ pollution sources and sinks in the United States.
 \end{abstract}

 \section*{Introduction}
  \subsection*{ASCENDS Mission}
   \paragraph{}
    The Active Sensing of Carbon Dioxide Emissions over Nights, Days, and Seasons (ASCENDS) mission, is currently an exploratory mission using simultaneous laser remote sensing of CO2 and O2. The technology employed is based off of a recommendation from the National Research Council of the United States National Academies. Active sensing will allow the ASCENDS team to enhance understanding of the role of CO2 in the global carbon cycle, due to its lack of seasonal, latitudinal, or diurnal bias.
  \subsection*{Visualization}
   \paragraph{}
    The ASCENDS Team wishes to visulaize the data they gather with their active sensing technology. Where as your average person will immediately tune out when confronted with a barrage of numbers, visualizations will allow said person to immediately have a basic understanding of the data. Additionally a visualization can present an enormous amount of data in an easily accessible format. As the ASCENDS technology is extremely accurate, a visualization can also present data from CO2 models to allow for a visual compare and contrast.
   \paragraph{}
    Generation of the visualizations is accomplished by first piecing the data together in a relational database using the Gigantron framework. Gigantron is an inhouse framework designed to ease the process of importing data, linking it together, and manipulating it. Gigantron runs on top of JRuby, a Ruby implementation on the Java Virtual Machine, for speed and rapid development. Rake tasks inside the generated Gigantron project allow for the dynamic generation of KML files, by utilizing ActiveRecord and ruby-kml.

  \subsection*{Importance}
   \paragraph{}
    The importance of this project lies in the ability of the visualization to be used to better understand $CO_2$ movement patterns and ultimately illustrate the need for an ASCENDS project satellite so that more data can be recorded and imported into the visualization, eventually allowing for a more global analysis of the atmospheric $CO_2$ levels. The data from the HYSPLIT model will allow for a back trajectory of the carbon dioxide plumes, so that their movement can be studied, and the in-situ data on top of the ITT data shows the carbon dioxide mixing ratio with the wind. The differences between the observed data and the predicted data will allow for a very hi-resolution analysis of the carbon dioxide levels. We believe that the predictive models will very closely resemble the observed data, thus we theorize that the US will be a net source of $CO_2$.
 \section*{Methodology}
  \subsection*{Data Obtainment}
   \paragraph{}
    The first step of the project was obtaining the raw data from the ASCENDS team’s various flights over eastern Virginia and North Carolina in a plane outfitted with $CO_2$ sensing equipment. The data was provided in multiple spreadsheets and raw text files, and included information about the readings such as wind speed, direction, latitude, longitude, $CO_2$, and date. Time stamps connected the data across multiple files. One set of data contains the ITT information, which is the average $CO_2$ level over the column of air below the plane. The second set of data is the in-situ data, which is the mixing ratio of $CO_2$ as compared to the rest of the air.
  \subsection*{Data Visualization}
   \paragraph{}
    After the data files were obtained, the files had to be parsed to retrieve the essential information. Scripts written in the JRuby programming language (a Java implementation of the Ruby interpreter) were used to analyze and parse the files. The resulting data was then written into a KML file using the ruby-kml library. The algorithms in the program were used to display the ITT data in column format, with the in-situ data on top of the ITT columns. A separate algorithm was used to color-code the resulting polygonal columns according to the level of $CO_2$ in the air, using the established ASCENDS color bar. The resulting KML file can then be imported into Google Earth, where the latitude and longitude data is used to map out the columns on the exact place where the data points were gathered. This allows the data to be analyzed in a geographical and visual sense.
  \subsection*{Model Visualization and Comparison}
   \paragraph{}
    Data from the various predictive models was then written into KML files that were also imported into Google Earth so that they can be compared to the observed data. A cross-sectional analysis of the observed data versus the predictive data will be visualized based on both the time and the location of the data point.
 \section*{Results}
  \subsection*{Visualization}
  \subsection*{Carbon Levels}

 \section*{Discussion}
  \paragraph{}
   The carbon dioxide in the atmosphere can now be visualized according to their exact geographic location. This allows both scientists and politicians to better interpret the visualized data instead of the data being nothing more than page after page of numbers and readings. The display of the in-situ mixing ratio data alongside the ITT data will allow for a more in-depth analysis of how carbon dioxide levels affect the composition of the air in specific locations in the atmosphere. This ASCENDS team visualization can also be used to calibrate future visualizations and serve as a benchmark for $CO_2$ analysis in further projects.
 \section*{Conclusion}

 \section*{Acknoledgements}

\end{document}


